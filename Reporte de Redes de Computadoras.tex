\documentclass{article}

\usepackage{lmodern}
\usepackage[T1]{fontenc}
\usepackage[spanish,activeacute]{babel}
\usepackage{mathtools}
%\usepackage{fancyhdr}
\usepackage{geometry,pdfpages}

\geometry{
	letterpaper,%showframe,
	left=30mm,top=10mm,right=30mm,bottom=10mm,
	headheight=6mm,headsep=7mm,foot=5mm,footskip=15mm,
	includeheadfoot
}

\title{SITUACI\'ON ACTUAL DE UGANDA EN LA IMPLEMENTACI\'ON DE REDES DE COMPUTADORAS\\}
\author{German Alexander Castro Portillo (00229017)}
\date{Mayo 18 de 2019}
%\pagestyle{fancy}
%\fancyhead[L]{date}

\begin{document}

	\maketitle

	\section{Progreso en Uganda}
		\large{
			Uganda es un pa'is que debido a la dificultad de poder invertir en nuevas tecnologias no se encuentra un desarrollo constante en el 'area de las redes de computadoras. En 2017 seg'un ``internet world stats`` hay aproximadamente 13 millones de usuarios de internet en Uganda lo que representa un $32\%$ de la población. El primer paso para alcanzar este logro se dio gracias a que, pese a la brecha existente, el progreso en Uganda es apoyado fuertemente por empresas extrangeras, en este caso fue Google quien en Diciembre de 2015 a trav'es de su proyecto ``Project Link`` lanz'o la primera conexi'on Wi-fi en la capital de Uganda Kampala apoyando tanto a proveedores m'oviles como a proveedores de internet para multiplicar los 'indices de transmisi'on de datos.    
		}
	\section{Situaci'on actual en Uganda}
		\large{
			Analizando las cifras provistas por ``internet world stats`` de 2017 y comparandolas con las cifras actuales podemos observar una gran decaida de usuarios de internet en 2019. Debido a que el gobierno de uganda impueso una tasa sobre el uso de redes OTT (redes sociales) Uganda perd'io un aproximado de 5 millones de usuarios en toda la regi'on seg'un datos obtenidos por la ``Comisión de Comunicaciones de Uganda (UCC)`` esto causo un gran impacto negativo no solo a nivel economico y social sino tambi'en en el 'area de las redes de computadoras ya que este decremento de usuarios afecto a los avances que se estaban realizando y interfiero en la ayuda que empresas extranjeras estaban brindandole al pa'is.
		}
	\section{Conclusi'on}
		\large{
			La tendencia de uso de internet en Uganda desde inicios del 2000 era alcista pues cada año hab'ia un incremento de usuarios del $47\%$ pero en la actualidad, como se menciono anteriormente, ha tenido una gran decaida que no solo afecto en lo economico al pa'is sino tambi'en en el progreso de las redes de computadoras. Debido a este aumento en los impuestos y decremento de usuarios en general se ha abierto a'un m'as la brecha digital ya existente y ha generado a su vez una gran brecha de seguridad, ya que el grupo de usuarios que siguen haciendo uso de la red se divide en dos: aquellos que han decidido pagar el impuesto y aquellos que han optado por utilizar aplicaciones VPN. Esto ha causado que la seguridad informatica de los usuarios pertenecientes al segundo grupo aumente pues exponen su informaci'on personal con quien administra las VPN. Mientras Uganda continue aplicando este impuesto se espera que las brechas anteriormente mencionadas sigan aumentando.
		}
\end{document}