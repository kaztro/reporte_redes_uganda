\documentclass{article}

\usepackage{lmodern}
\usepackage[T1]{fontenc}
\usepackage[spanish,activeacute]{babel}
\usepackage{mathtools}
\usepackage{graphicx}
\usepackage{geometry,pdfpages}

\geometry{
	letterpaper,%showframe,
	left=30mm,top=10mm,right=30mm,bottom=10mm,
	headheight=6mm,headsep=7mm,foot=5mm,footskip=15mm,
	includeheadfoot
}

\begin{document}
	
	\begin{titlepage}
		\centering
		\includegraphics{logo}\par\vspace{0.2cm}
		{\scshape\Large UNIVERSIDAD CENTROAMERICANA JOS'E SIME'ON CA'NAS \par}
		\vspace{3cm}
		{\scshape\LARGE ''SITUACI\'ON ACTUAL DE UGANDA EN LA IMPLEMENTACI\'ON DE REDES DE COMPUTADORAS'' \par}
		\vspace{3cm}
		{\Large Catedr'atica: Elisa Cristina Aldana Calderon \par}
		\vspace{1cm}
		{\Large Estudiante: German Alexander Castro Portillo  \par}
		\vfill
		{\large 18 de Mayo de 2019 \par}
	\end{titlepage}
	
	\section*{Introducci'on}
		\large{
			Uganda es un pa'is que, as'i como muchos otros dentro de 'Africa, ha demostrado desde inicios de los 2000 querer progresar lo que con el pasar del tiempo lo convirtio en un ejemplo perfecto de que si se posee la ayuda necesaria s'i se puede progresar, pero as'i como ha sido un buen ejemplo de progreso dentro de los paises 'Africanos tambi'en en los 'ultimos a'nos lastimosamente tambi'en ha demostrado que con una administraci'on no adecuada y una mala gesti'on de recursos se pueden perder a'nos de progreso.  
		}
	
	\section{Progreso en Uganda}
		\large{
			Uganda es un pa'is que debido a la dificultad de poder invertir en nuevas tecnologias no se encuentra un desarrollo constante en el 'area de las redes de computadoras. En 2017 seg'un ``internet world stats`` hay aproximadamente 13 millones de usuarios de internet en Uganda lo que representa un $32\%$ de la población [1]. El primer paso para alcanzar este logro se dio gracias a que, pese a la brecha existente, el progreso en Uganda es apoyado fuertemente por empresas extrangeras, en este caso fue Google [2] quien en Diciembre de 2015 a trav'es de su proyecto ``Project Link`` lanz'o la primera conexi'on Wi-fi en la capital de Uganda Kampala apoyando tanto a proveedores m'oviles como a proveedores de internet para multiplicar los 'indices de transmisi'on de datos.    
		}
	\section{Progreso en Uganda}
		\large{
			Analizando las cifras provistas por ``internet world stats`` de 2017 y comparandolas con las cifras actuales podemos observar una gran decaida de usuarios de internet en 2019 [3]. Debido a que el gobierno de uganda impueso una tasa sobre el uso de redes OTT (redes sociales) Uganda perd'io un aproximado de 5 millones de usuarios en toda la regi'on seg'un datos obtenidos por la ``Comisión de Comunicaciones de Uganda (UCC)`` [4] esto causo un gran impacto negativo no solo a nivel economico y social sino tambi'en en el 'area de las redes de computadoras ya que este decremento de usuarios afecto a los avances que se estaban realizando y interfiero en la ayuda que empresas extranjeras estaban brindandole al pa'is.
		}
	\section{Situaci'on actual de Uganda}
		\large{
			La tendencia de uso de internet en Uganda desde inicios del 2000 era alcista pues cada año hab'ia un incremento de usuarios del $47\%$ [5] pero en la actualidad, como se menciono anteriormente, ha tenido una gran decaida que no solo afecto en lo economico al pa'is sino tambi'en en el progreso de las redes de computadoras. Debido a este aumento en los impuestos y decremento de usuarios en general se ha abierto a'un m'as la brecha digital ya existente y ha generado a su vez una gran brecha de seguridad, ya que el grupo de usuarios que siguen haciendo uso de la red se divide en dos: aquellos que han decidido pagar el impuesto y aquellos que han optado por utilizar aplicaciones VPN [6] . Esto ha causado que la seguridad informatica de los usuarios pertenecientes al segundo grupo aumente pues exponen su informaci'on personal con quien administra las VPN. Mientras Uganda continue aplicando este impuesto se espera que las brechas anteriormente mencionadas sigan aumentando.
		}
	\begin{thebibliography}{00}
		\bibitem{1} https://www.internetworldstats.com/af/ug.htm  
		\bibitem{2} https://www.bbc.com/news/technology-35000544 
		\bibitem{3} https://www.trecebits.com/2019/02/19/uganda-pierde-5-millones-de-usuarios-de-internet-al-cobrar-por-usar-las-redes-sociales/
		\bibitem{4} https://www.iafrikan.com/2019/02/09/a-drop-in-internet-users-in-uganda-as-a-result-of-social-media-tax/
		\bibitem{5} https://www.efe.com/efe/america/tecnologia/las-redes-sociales-en-uganda-dejan-de-ser-gratuitas-desde-hoy/20000036-3667953
		\bibitem{6} https://www.genbeta.com/actualidad/uganda-puso-impuesto-a-redes-sociales-entonces-3-millones-personas-han-dejado-usar-internet
	\end{thebibliography}
\end{document}